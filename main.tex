\documentclass[a4j]{jarticle}
\usepackage{ascmac}
\usepackage{graphicx}

\def\year{2021}
\def\group{Advanced Systems Group}

\title{\gt ゼミの心得}
\author{山田 浩史 \\ The Principal Investigator of Advanced Systems Group}
\date{}
\begin{document}
\maketitle

本稿では,Advanced Systems Group において実施しているゼミについて紹介する.具体的にはゼミの形式および心構えについて説明する.私が効率厨であるため,どうせやるならゼミを限界まで有意義なものにしようという思いが強い.ぜひとも一読いただきたい.

\section*{ゼミの形式}
ゼミの基本形は,2 名が話者となりそれぞれがプレゼンをして他のメンバーは聴衆となってその内容を理解したり質問したりする.話者はスライドを作成し,他のメンバーの前で発表する.だいたい 18〜20 枚ぐらいのスライドにまとめて話すことが多い.集中力および生産性を最大化するために,一切の内職を禁止し,長くても 1.5 時間で終わらせる.

本研究室のゼミは2種類に分けられる.

\begin{itemize}
 \item \textbf{教科書ゼミ:} B4 の学生さんが前期に実施するゼミである.英語の教科書を読んでその内容を理解し,スライドにまとめ,発表するというものである.
 \item \textbf{論文ゼミ:} B4 後期以降のゼミである.自身の研究と関連する論文を読んでその内容を理解し,スライドにまとめ,発表するというものである.
\end{itemize}

ここで進捗ゼミなるものがないことに注意されたい.多くの研究室が各個人の研究の進捗を報告するゼミを実施しているが,本研究室ではそのようなゼミは設けないことにしている.他人の進捗を聞いたところで,聴衆にとって得るものは少ない,もしくは何も無いためである.進捗は個別の打ち合わせで山田は把握しているし,研究室内で近い研究テーマを持つ学生さん同士のノウハウ共有は wiki を用意しているのでそこを利用できる.

論文ゼミは,学生さんだけでなく山田も参加する.1人でも多く論文ゼミを実施した方が多くの知識を得られるためである.また,学生さんの研究テーマと離れた論文を紹介することで,システムソフトウェア研究の最前線を知ることができるためである.

\section*{ゼミの心構え}
ゼミ参加者となる話者と聴衆の基本的な心構えは\emph{「話者と聴衆の両方に利益のある時間を」}であり,皆の貴重な時間を割いて実施するこのイベントを大事に使いたい.ゼミは,話者には最先端文献の読解力向上および英語・プレゼン能力の開発,聴衆には知識の拡張の機会を与える.ゼミが有意義になるかは気持ちの持ちようであり,きちんとやった分だけ自分の能力向上に繋がるのでそのための努力は惜しまないようにしたい.

具体的には以下の点を意識することで,ゼミを実りのあるイベントにすることができ,自身の能力開発に繋がる.

\begin{itemize}
  \item \textbf{文献の著者の立場でプレゼンをする:} スライドの文面・掲載していることには話者が責任を持ち,質問されていてもきちんと答えられるようにする.質問した際に「こう書いてありました」という応答はあってはならない.それでは誰も意味がわからず,無駄な時間を過ごしてしまう.事前にきちんと調査する,調査してもわからない場合には誰かに聞くなどをして,自分のスライドにしてからプレゼンに望むことが必須である.

  \item \textbf{理解した内容でスライドを作る:} 教科書や論文の構成どおりにプレゼンをするのではなく,プレゼン用に構成をしなおす.文献の内容をきちんと理解した上で,文章や図を自分なりに作る必要がある.読み物としてのまとめ方とプレゼンとしてのまとめ方は異なる.読み物は時間をかけてじっくり読んだり読み返しができるので,1 つのセンテンスや図の情報量が多い.一方で,プレゼンは理解に時間をかけられないので,聴衆の直感に合う構成および文と図に簡潔さが求められる.

  \item \textbf{よいスライドはパクる:} 他人のスライドで良い部分があれば積極的にパクるようにする.まとめ方,情報の並べ方,図の作り方など,良い点は自分のスライドやトークに取り入れていく.

  \item \textbf{色々試してみる:} 図やアイテマイズの文面を自身で考え,試し切りの場としてゼミ使って欲しい.試行錯誤すればするほど自身の能力向上に繋がる.

  \item \textbf{わからないことは質問する:} 研究室内で実施するので,わからないことは遠慮なく聞く.

\end{itemize}


\section*{論文ゼミテンプレ攻略法}

以下に研究プレゼンの構成テンプレートを示す.論文ゼミだけでなく,自身の研究発表や論文購読にも使えるものである.端的に言うと,研究発表の聴衆が聞きたいであろうことを逆算した形の構成となっている.システムソフトウェア研究の発表において聞きたいことは,どのような学術的な貢献のある研究であるかという点である.噛み砕くと,どのような問題意識で,どのような提案をして,どのように解決しているのか,という点を聞きたいのである.聞き手はこれらの点が妥当であるか,興味深いかを頭の中で整頓しながら聴講する.この情報をプレゼンすればよい.ただ,このテンプレートに当てはめる際には,論文もしくは自身の研究が,どのような問題意識で,どのような提案をして,どのように解決しているのか,というのを徹底的に理解する必要があることに注意されたい.

\begin{enumerate}
 \item \textbf{Background(1・2枚):} 研究が題材としている状況を説明する.前提としている環境や対象とするソフトウェアの説明が一般的となる.仮想マシン移送の高速化の研究であれば,仮想マシン移送を,聴衆によっては仮想マシンモニタから紹介すればよい.

 \item \textbf{Motivation(1・2枚):} なぜ提案方式が必要であるかを説明する.背景を受けて,どのような問題点があるのか,既存技術では何が不足しているのか,を説明するのが一般的である.仮想マシン移送の高速化の研究であれば,なぜ移送は遅いのか,既存の高速化技術のどこに問題があるのかを言えばよい.

 \item \textbf{Proposal(1枚):} どのような効果があるものを提案するのかを説明する.どのようにその効果を達成するかはアプローチに相当するので,それは次のスライドに取っておく.1つ前のスライドで指摘する事項を受けて,達成するべき目標を言う.仮想マシン移送の高速化の研究であれば,既存技術の問題点として挙げた事柄の逆を言えばよい.
   
 \item \textbf{Approach(1枚):} 提案をどのように実現するかを説明する.いわば提案方式の核となる技術的アイディアを言う.そのアイディアで,前のスライドで挙げた目標がどのように達成されるかを\emph{一言}で示す.仮想マシン移送の高速化の研究であれば,どう高速化するのかを言えばよい.

 \item \textbf{Technical Challenges(1枚):} アプローチを実現する上での難しさを説明する.次に説明する Design Details の前フリに相当し,どこが技術的なポイントなのかを際立たせる.仮想マシン移送の高速化の研究であれば,高速化方式を実現する上で特殊なことをした部分,その理由を言えばよい.

 \item \textbf{Design Details(3・4枚):} 技術的な詳細を説明する.Technical Challenges を受けて,どのような技術でその難しさを克服したのかを説明する.仮想マシン移送の高速化の研究であれば,提案する高速化方式の技術的なポイントを言えばよい.

 \item \textbf{Experiments(5・6枚):} 実験を説明する.テンプレートとしては,実験の説明で 1 枚,その結果で 1 枚という 2 枚 1 組で考える.実験の説明では,その実験の目的,方法を説明する.結果はグラフや表を掲載し,何が結論づけられるかを説明する.提案で掲げていることが満たされているかを確認できる実験を取り上げる.仮想マシン移送の高速化の研究であれば,いかに高速化されたのか,それに伴うコストは何かがわかる実験を言えばよい.

 \item \textbf{Conclusions(1枚):} 発表をまとめる.何が動機で何を提案し,実験から何が言えるかを説明する.

\end{enumerate}

テンプレートに当てはめるだけでなく,なぜこのテンプレートでよいのかを常に考えることで応用が効いてくる.上記テンプレートは,新技術の提案系の研究に当てはめるものであるが,たとえば,システムソフトウェアの評価系の研究も,大枠はこのテンプレートに当てはめることができる.Approach に実験方法を,それ以降に各種評価実験をまとめると言った具合にまとめられる.こうすれば,きちんと研究内容を伝えられる.

少し脱線するが,最初はテンプレートなどの型を学び,そのあとに自分の好きなスタイルでやりたいようにやればよい.歌舞伎界に「型があるから型破り.型が無ければ,それは形無し.」という言葉がある.基礎・基本がある(型あり)からこそ正しい方向に発展させていける(型破り)のであり,基礎・基本なくして(型無し)は何もできないのである.型無しで何か物事をやってうまくいくパターンがあるかもしれないが,それはたまたまであり継続して成功を呼ぶことができない.確固たる能力を身につけるには,まずは型ありの状態になることが大切なのである.

\end{document}
